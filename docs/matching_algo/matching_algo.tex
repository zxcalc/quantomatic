\documentclass{article}
\usepackage{fullpage}
\usepackage{graphicx}

\begin{document}

\section{Match State}

For a string graph $G$, let $N(G)$ be the set of node-vertices and $W(G)$ the wire-vertices.

Let $L$ be the pattern and $G$ the target. A match state consists of:

\begin{itemize}
    \item $U_c \subseteq W(L)$ : a set of unmatched circles
    \item $U_w \subseteq W(L)$ : a set of unmatched (non-circle) wire-vertices
    \item $U_n \subseteq N(L)$ : a set of unmatched node-vertices
    \item $P \subseteq N(L)$ : a set of partially-matched node-vertices
    \item $P_s \subseteq P$ : a set of partially-matched node-vertices scheduled for re-checking
    \item $m : V_L \rightarrow V_G$ : a partial injective function of the current match. When this is total and local iso, this match is complete.
\end{itemize}

\section{Inner Loop: Concrete Matching}

The goal of the concrete matching step is to match all of the concrete vertices in the graph. If a node-vertex is not connected to a !-box, its complete neighbourhood will also be matched. If it is connected to one or more !-boxes, only some if its neighbourhood is matched. We refer to these as \textit{partially matched} node-vertices. A node-vertex $v$ such that the neighbourhood of $m(v)$ is totally covered is called \textit{completely matched}.

\begin{center}
    \includegraphics[height=16cm]{concrete_part.pdf}
\end{center}

\section{Outer Loop: Pattern Expansion}

\begin{center}
    \includegraphics[height=16cm]{pattern_part.pdf}
\end{center}



\end{document}