\documentclass{article}
\usepackage{amsmath}
\usepackage{amssymb}
\usepackage{fullpage}
\usepackage{graphicx}

\DeclareMathOperator{\dom}{dom}
\DeclareMathOperator{\codom}{cod}

\begin{document}

\section{Normalisation}

We normalise both the pattern and target graphs by ensuring that
\begin{itemize}
\item each wire connected to a node-vertex at both ends has exactly two wire-vertices on it
\item boundary vertices are adjacent to either node-vertices or another boundary vertex
\item circles (wires in a loop with no node-vertices) have exactly two wire-vertices on them
\end{itemize}

\section{Match State}

For a string graph $G$, let $N(G)$ be the set of node-vertices and $W(G)$ the wire-vertices.

Let $L$ be the pattern and $G$ the target. A match state consists of:

\begin{itemize}
    \item $U_c \subseteq W(L)$ : a set of unmatched circles
    \item $U_w \subseteq W(L)$ : a set of unmatched (non-circle) wire-vertices
    \item $U_n \subseteq N(L)$ : a set of unmatched node-vertices
    \item $P \subseteq N(L)$ : a set of partially-matched node-vertices
    \item $P_s \subseteq P$ : a set of partially-matched node-vertices scheduled for re-checking
    \item $m : V_L \rightarrow V_G$ : a partial injective function of the current match. When this is total and local iso, this match is complete.
\end{itemize}

Note that circles are given as the name of smallest wire-vertex in the circle, w.r.t. the name ordering.

\section{Inner Loop: Concrete Matching}

The goal of the concrete matching step is to match all of the concrete vertices in the graph. If a node-vertex is not connected to a !-box, its complete neighbourhood will also be matched. If it is connected to one or more !-boxes, only some if its neighbourhood is matched. We refer to these as \textit{partially matched} node-vertices. A node-vertex $v$ such that the neighbourhood of $m(v)$ is totally covered is called \textit{completely matched}.

The concrete matching routine is given a function ``doomed'', which returns true if a match of a concrete node-vertex is impossible. For !-box matching this returns true if the (concrete) neighbourhood of $v$ contains fewer in-edges (resp. out-edges) than that of $m(v)$ and there are no !-boxes connected to $v$ by an in-edge (resp. out-edge). It also takes a match state, subject to the following preconditions:

\subsection{Preconditions}
\begin{itemize}
  \item $L$ and $G$ are both normalised
  \item $U_c\cap\dom(m) = U_w\cap\dom(m) = U_n\cap\dom(m) = \varnothing$ (ie: the vertices in the unmatched sets are not part of the existing matching)
  \item each vertex in $U_w$ is adjacent to something in $U_n\cup P$ (ensures we can always reach wire vertices in $U_w$ by starting from a node-vertex in $U_n$ or $P$)
  \item $P \subseteq \dom(m)$ ($P$ contains either partially-matched or completely-matched vertices)
  \item $U_c$ genuinely does contain circles
\end{itemize}

\begin{center}
  \includegraphics[height=16cm]{concrete_part.pdf}
\end{center}

\subsection{Postconditions}
These are true for the match state that results from each successful branch:
\begin{itemize}
  \item $U_c = U_w = U_n = P_s = \varnothing$ (every vertex marked for matching was handled)
  \item $m$ is a partial matching from $L$ to $G$ that satisfies local iso condition for concrete vertices not in $P$,
  \item $P$ is exactly the set of vertices in $\dom(m)$ that are adjacent to a vertex not in $\dom(m)$ ($P$ contains exactly the partially-matched vertices)
  \item $\dom(m)$ is the union of the initial states of $U_c$, $U_w$, $U_n$ and $\dom(m)$ (we have matched everything we were asked to, and not lost anything in the process)
\end{itemize}

\section{Testing Wrapper: Concrete Matching Only}

For testing purposes, it is straight-forward to wrap the above procedure to just do concrete matching.

\begin{center}
  \includegraphics[height=12cm]{concrete_wrapper.pdf}
\end{center}

\section{Full Wrapper: Lazy !-Box Expansion}

Since we are only matching !-graphs on to concrete graphs, it suffices to restrict !-box operations to only two: KILL and EXPAND (COPY, then DROP the new box).

\begin{center}
  \includegraphics[height=16cm]{pattern_part.pdf}
\end{center}



\end{document}
